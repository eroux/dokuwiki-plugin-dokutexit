\ifnum 0\ifxetex 1\fi\ifluatex 1\fi>0
%%
%% This section is for LuaTeX and XeTeX
%% Warning: polyglossia works with LuaTeX starting from TL2013
%%
  \usepackage{polyglossia}
  \setmainlanguage{english}
\else
%%
%% next section is for latex and pdflatex:
%%
  \usepackage[french]{babel} %% Use your own babel language
  \usepackage{ucs}
  \usepackage[utf8x]{inputenc}
  \usepackage[T1]{fontenc}
  \usepackage{pslatex}
  \usepackage{wasysym}
  \usepackage{textcomp}
\fi
%%
%% and this section is common
%%
%% uncomment these if you use them
\usepackage{supertabular}  % very big tables
\usepackage{pdflscape}     % idem
\usepackage{lastpage}      % trick last page
\usepackage{eso-pic}       % if you want to put some text or image in the background
\usepackage{moreverb}      % for verbatimtab environment FIXME: is this really necessary?
% Set your margins here:
\usepackage[top=3cm,bottom=3cm,left=2.5cm,right=2.5cm]{geometry}
% If you want to have colored links in your pdf, change the following line:
\usepackage[unicode, bookmarks, colorlinks=false, pdfborder={0 0 0}]{hyperref}
\usepackage{graphicx}       % provides image insertion
\usepackage{tabularx}       % provides table insertion
\usepackage{color}          % provides color handling
\usepackage{listings}       % provides syntaxic coloration for code
\usepackage{fancyhdr}       % provides headings management
\usepackage[normalem]{ulem} % provides underlining (\uline{})
\usepackage{wrapfig}        % usefull for image 
